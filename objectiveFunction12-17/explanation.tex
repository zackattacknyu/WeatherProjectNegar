\documentclass[11pt,psfig]{article}
\usepackage{epsfig}
\usepackage{times}
\usepackage{amssymb}
\usepackage{float}

\newcount\refno\refno=1
\def\ref{\the\refno \global\advance\refno by 1}
\def\ux{\underline{x}}
\def\uw{\underline{w}}
\def\bw{\underline{w}}
\def\ut{\underline{\theta}}
\def\umu{\underline{\mu}} 
\def\bmu{\underline{\mu}} 
\def\be{p_e^*}
\newcount\eqnumber\eqnumber=1
\def\eq{\the \eqnumber \global\advance\eqnumber by 1}
\def\eqs{\eq}
\def\eqn{\eqno(\eq)}

 \pagestyle{empty}
\def\baselinestretch{1.1}
\topmargin1in \headsep0.3in
\topmargin0in \oddsidemargin0in \textwidth6.5in \textheight8.5in
\begin{document}
\setlength{\parskip}{1.2ex plus0.3ex minus 0.3ex}


\thispagestyle{empty} \pagestyle{myheadings} \markright{G}



In order to sort the prediction pairs, I will start with the question of what is the gap between the error function values for each function and how do they compare. \\
\\
Let $f$ be the first error function.\\
Let $g$ be the second error function.\\
Let $X$ be our first patch and $Y$ be our second patch.\\
\\
We will look at the pairs which disagree, so without loss of generality\\
$f(X) < f(Y)$ and $g(X) > g(Y)$ \\
\\
In order to say that $f$ is good, we would want to minimize the following
\[
(f(Y)-f(X)) - \alpha(g(X)-g(Y))
\]
This way the lowest values occur when the gap between the $f$ values is small but the gap between the $g$ values is large.\\
\\
We also care more about the cases where the minimum $f$ value is small since if the above $f$ and $g$ values are large then they are just both saying they are bad and our function is not as conclusive. \\
\\
Along the same vain, we also care about cases where the minimum $g$ values is large.\\
\\
Our function now becomes the following:
\[
(f(Y)-f(X)) - \alpha(g(X)-g(Y)) + \beta f(X) - \gamma g(Y)
\]
I will try both raw values and percentile ranks.

\end{document}








