%%
%% Class homework & solution template for latex
%% Alex Ihler
%%
\documentclass[twoside,11pt]{article}
\usepackage{amsmath,amsfonts,amssymb,amsthm}
\usepackage{graphicx,color}
\usepackage{verbatim,url}
\usepackage{listings}
\usepackage{upquote}
\usepackage[T1]{fontenc}
%\usepackage{lmodern}
\usepackage[scaled]{beramono}
%\usepackage{textcomp}

% Directories for other source files and images
\newcommand{\bibtexdir}{../bib}
\newcommand{\figdir}{fig}

\newcommand{\E}{\mathrm{E}}
\newcommand{\Var}{\mathrm{Var}}
\newcommand{\N}{\mathcal{N}}
\newcommand{\matlab}{{\sc Matlab}\ }

\setlength{\textheight}{9in} \setlength{\textwidth}{6.5in}
\setlength{\oddsidemargin}{-.25in}  % Centers text.
\setlength{\evensidemargin}{-.25in} %
\setlength{\topmargin}{0in} %
\setlength{\headheight}{0in} %
\setlength{\headsep}{0in} %

\renewcommand{\labelenumi}{(\alph{enumi})}
\renewcommand{\labelenumii}{(\arabic{enumii})}

\theoremstyle{definition}
\newtheorem{MatEx}{M{\scriptsize{ATLAB}} Usage Example}

\definecolor{comments}{rgb}{0,.5,0}
\definecolor{backgnd}{rgb}{.95,.95,.95}
\definecolor{string}{rgb}{.2,.2,.2}
\lstset{language=Matlab}
\lstset{basicstyle=\small\ttfamily,
        mathescape=true,
        emptylines=1, showlines=true,
        backgroundcolor=\color{backgnd},
        commentstyle=\color{comments}\ttfamily, %\rmfamily,
        stringstyle=\color{string}\ttfamily,
        keywordstyle=\ttfamily, %\normalfont,
        showstringspaces=false}
\newcommand{\matp}{\mathbf{\gg}}




\begin{document}

We want to use Earth-Movers Distance as a loss function instead of Mean Squared Error in order to better capture differences in structure between weather maps. Here is an example with 1D arrays of precipitation values. \\
\\
Suppose our target array $Y$ is the following:
\[
[2,5,0,0,0,0]
\]
We have the following two proposed predictions $\hat{Y_1}$ and $\hat{Y_2}$
\[
\hat{Y_1} =[0,0,2,5,0,0]
\]
\[
\hat{Y_2} = [0,0,0,0,2,5]
\]
Now suppose we have to pick the best prediction. The mean squared error between $\hat{Y_1}$ and $Y$ is the same as the mean squared error between  $\hat{Y_2}$ and $Y$. However, it is clear that $Y_1$ is better since the two precipitation amounts were shifted less. My function attempts to correct for this by calculating the work it takes to transform the prediction into the target. \\
\\
Currently, we calculate this function on 20 by 20 patches of precipitation amounts. Here are the details on how the function obtains a loss amount between patches.\\
\\
Let the target patch have pixels $x_{ij}$\\
\[
\begin{bmatrix}
    x_{11} & x_{12} & x_{13} & \dots  & x_{1n} \\
    x_{21} & x_{22} & x_{23} & \dots  & x_{2n} \\
    \vdots & \vdots & \vdots & \ddots & \vdots \\
    x_{m1} & x_{m2} & x_{m3} & \dots  & x_{mn}
\end{bmatrix}
\]
Let the prediction patch have pixels $\hat{x_{kl}}$\\
\[
\begin{bmatrix}
    \hat{x_{11}} & \hat{x_{12}} & \hat{x_{13}} & \dots  & \hat{x_{1n}} \\
    \hat{x_{21}} & \hat{x_{22}} & \hat{x_{23}} & \dots  & \hat{x_{2n}} \\
    \vdots & \vdots & \vdots & \ddots & \vdots \\
    \hat{x_{m1}} & \hat{x_{m2}} & \hat{x_{m3}} & \dots  & \hat{x_{mn}}
\end{bmatrix}
\]
Without loss of generality, 
\[
\sum_{kl}{\hat{x}_{kl}} \leq \sum_{ij}{x_{ij}}
\]
We are going to calculate a measure of work to transform the prediction into the target. In order to accomplish that, we need to know the pairwise distance between pixels. We will thus denote $c_{ijkl}$ as the distance between $ij$ and $kl$ pixels thus
\[
c_{ijkl} = \sqrt{(i-k)^2 + (j-l)^2}
\]
Let $f$ be a flow vector where entry $f_{ijkl}$ represents the flow from pixel $x_{ij}$ to pixel $\hat{x_{kl}}$\\
\[
WORK = \min_f(\sum_{ij}{\sum_{kl}{c_{ijkl} f_{ijkl}}} + \alpha \sum_{ij}{(x_{ij} - \sum_{kl}{f_{ijkl}})^2})
\]
subject to the following constraints:\\
\[
f_{ijkl} \geq 0 \, \, \forall ijkl
\]
\[
\sum_{ij}{f_{ijkl}} = \hat{x}_{kl} \, \, \forall kl
\]
\[
\sum_{kl}{f_{ijkl}} \leq x_{ij} \, \, \forall ij
\]
Earth-Movers Distance is defined as follows:
\[
EMD = \frac{WORK}{\sum_{kl}{\hat{x}_{kl}}}
\]


\end{document}
