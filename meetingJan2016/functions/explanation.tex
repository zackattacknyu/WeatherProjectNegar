\documentclass[12pt,psfig]{article}
\usepackage{epsfig}
\usepackage{times}
\usepackage{amssymb}
\usepackage{float}

\newcount\refno\refno=1
\def\ref{\the\refno \global\advance\refno by 1}
\def\ux{\underline{x}}
\def\uw{\underline{w}}
\def\bw{\underline{w}}
\def\ut{\underline{\theta}}
\def\umu{\underline{\mu}} 
\def\bmu{\underline{\mu}} 
\def\be{p_e^*}
\newcount\eqnumber\eqnumber=1
\def\eq{\the \eqnumber \global\advance\eqnumber by 1}
\def\eqs{\eq}
\def\eqn{\eqno(\eq)}

 \pagestyle{empty}
\def\baselinestretch{1.1}
\topmargin1in \headsep0.3in
\topmargin0in \oddsidemargin0in \textwidth6.5in \textheight8.5in
\begin{document}
\setlength{\parskip}{1.2ex plus0.3ex minus 0.3ex}


\thispagestyle{empty} \pagestyle{myheadings} \markright{G}

Let $f$ be our EMD function\\
Let $g$ be the MSE\\
Let $X$ be the patch chosen by EMD \\
and $Y$ be patch chosen by MSE\\
\\
We thus know that $f(X)<f(Y)$ and $g(Y)<g(X)$\\
\\
Let $H_1=f(Y)-f(X)$ and $H_2=g(X)-g(Y)$\\
\\
Assumption: 
Let $H=H_1 \cdot H_2$

\newpage

Let $f$ be the error function which we are trying to prove is the best one.\\
Let $g$ be the other error function.\\
Let $p(A,B)$ be the probability of choosing patch A over patch B.\\
Let $X$ and $Y$ be our two patches.\\
\\
We will focus on examples where $f$ and $g$ predict different patches, \\
thus without loss of generality,\\
$f(X) < f(Y)$ and $g(X) > g(Y)$\\
\\
Let $H_1 = f(Y)-f(X)$ and $H_2 = g(X)-g(Y)$\\
Let $H = H_1 + \alpha H_2$ ($\alpha$ will be chosen so that $H_1$ and $\alpha H_2$ have the same range)\\
\\
Assumption 1: If $f$ is a good error function, then the $H_1$ value is correlated with our confidence that $X$ is a better patch than $Y$.\\
\\
Using assumption 1, I will sort the examples by $H_1$ in descending order. The probability of choosing $X$ over $Y$ should be very high in the initial examples and decrease as you progress through the examples. Thus $H_1$ and $p(X,Y)$ should be well correlated.\\
\\
I will then sort by $H_2$ in descending order and show the examples. If $f$ outperforms $g$ then we should find that $p(Y,X)$ and $H_2$ are not as well correlated as $p(X,Y)$ and $H_1$.  \\
\\
Assumption 2: If $f$ is a better error function than $g$, then $X$ should be better than $Y$ for the examples where they disagree the most.\\
\\
To formalize assumption 2, I will sort the examples by descending order of $H$ value. For the first $m$ examples, there should be a very high probability of choosing $X$ over $Y$. I will then flip the definition of $f$ and $g$ and show examples in descending order of $H$ value. The average probability of choosing $X$ over $Y$ for these first $m$ examples should be lower than previously.

\end{document}








