%%
%% Class homework & solution template for latex
%% Alex Ihler
%%
\documentclass[twoside,11pt]{article}
\usepackage{amsmath,amsfonts,amssymb,amsthm}
\usepackage{graphicx,color}
\usepackage{verbatim,url}
\usepackage{listings}
\usepackage{hyperref}
\usepackage{upquote}
\usepackage[T1]{fontenc}
%\usepackage{lmodern}
\usepackage[scaled]{beramono}
%\usepackage{textcomp}

% Directories for other source files and images
\newcommand{\bibtexdir}{../bib}
\newcommand{\figdir}{fig}

\newcommand{\E}{\mathrm{E}}
\newcommand{\Var}{\mathrm{Var}}
\newcommand{\N}{\mathcal{N}}
\newcommand{\matlab}{{\sc Matlab}\ }

\setlength{\textheight}{9in} \setlength{\textwidth}{6.5in}
\setlength{\oddsidemargin}{-.25in}  % Centers text.
\setlength{\evensidemargin}{-.25in} %
\setlength{\topmargin}{0in} %
\setlength{\headheight}{0in} %
\setlength{\headsep}{0in} %

\renewcommand{\labelenumi}{(\alph{enumi})}
\renewcommand{\labelenumii}{(\arabic{enumii})}

\theoremstyle{definition}
\newtheorem{MatEx}{M{\scriptsize{ATLAB}} Usage Example}

\definecolor{comments}{rgb}{0,.5,0}
\definecolor{backgnd}{rgb}{.95,.95,.95}
\definecolor{string}{rgb}{.2,.2,.2}
\lstset{language=Matlab}
\lstset{basicstyle=\small\ttfamily,
        mathescape=true,
        emptylines=1, showlines=true,
        backgroundcolor=\color{backgnd},
        commentstyle=\color{comments}\ttfamily, %\rmfamily,
        stringstyle=\color{string}\ttfamily,
        keywordstyle=\ttfamily, %\normalfont,
        showstringspaces=false}
\newcommand{\matp}{\mathbf{\gg}}




\begin{document}

I will begin with a sampling of patches that covers as many areas with precipitation as possible in the target set\\
\\
The target patches will be denoted $\{q_i\}$ and the corresponding prediction patches will be $\{\hat{q}_i\}$\\
\\
Due to the evidence $\href{http://www.ariel.ac.il/sites/ofirpele/publications/ICCV2009.pdf}{here}$ which says that thresholding distance provides an accurate and fast approximation, I am making the assumption that EMD, denoted as $g$, can be approximated as follows:
\[
\frac{1}{N} \sum_{i=1}^N{g(q_i,\hat{q}_i)}
\]
To set this up as an expectation, I will organize the patches by their sum. I will then organize the sum values into K bins. Let $p(q_i)$ be the probability that patch $q_i$ is in its current bin. Thus the above sum will be
\[
\frac{\sum_{i=1}^N{p(q_i) g(q_i,\hat{q}_i)}}{\sum_{i=1}^N{p(q_i)}} 
\]
I will then subsample $M$ patches out of the total $N$ that exist and compute the following
\[
\frac{\sum_{j=1}^M{p(q_j) g(q_j,\hat{q}_j)}}{\sum_{j=1}^M{p(q_j)}} 
\]

\end{document}
