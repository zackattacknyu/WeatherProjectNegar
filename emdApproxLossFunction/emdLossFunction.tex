%%
%% Class homework & solution template for latex
%% Alex Ihler
%%
\documentclass[twoside,11pt]{article}
\usepackage{amsmath,amsfonts,amssymb,amsthm}
\usepackage{graphicx,color}
\usepackage{verbatim,url}
\usepackage{listings}
\usepackage{hyperref}
\usepackage{upquote}
\usepackage[T1]{fontenc}
%\usepackage{lmodern}
\usepackage[scaled]{beramono}
%\usepackage{textcomp}

% Directories for other source files and images
\newcommand{\bibtexdir}{../bib}
\newcommand{\figdir}{fig}

\newcommand{\E}{\mathrm{E}}
\newcommand{\Var}{\mathrm{Var}}
\newcommand{\N}{\mathcal{N}}
\newcommand{\matlab}{{\sc Matlab}\ }

\setlength{\textheight}{9in} \setlength{\textwidth}{6.5in}
\setlength{\oddsidemargin}{-.25in}  % Centers text.
\setlength{\evensidemargin}{-.25in} %
\setlength{\topmargin}{0in} %
\setlength{\headheight}{0in} %
\setlength{\headsep}{0in} %

\renewcommand{\labelenumi}{(\alph{enumi})}
\renewcommand{\labelenumii}{(\arabic{enumii})}

\theoremstyle{definition}
\newtheorem{MatEx}{M{\scriptsize{ATLAB}} Usage Example}

\definecolor{comments}{rgb}{0,.5,0}
\definecolor{backgnd}{rgb}{.95,.95,.95}
\definecolor{string}{rgb}{.2,.2,.2}
\lstset{language=Matlab}
\lstset{basicstyle=\small\ttfamily,
        mathescape=true,
        emptylines=1, showlines=true,
        backgroundcolor=\color{backgnd},
        commentstyle=\color{comments}\ttfamily, %\rmfamily,
        stringstyle=\color{string}\ttfamily,
        keywordstyle=\ttfamily, %\normalfont,
        showstringspaces=false}
\newcommand{\matp}{\mathbf{\gg}}




\begin{document}

Let $P$ be a sampling of patches that covers as many areas with precipitation as possible in the target set. \\
\\
Let $N$ be the number of patches in $P$\\
Let $I$ be the numbers 1 to $N$\\
\\
Let $\{q_i\}_{i \in I}$ be target patches\\
Let $\{\hat{q}_i\}_{i \in I}$ be the corresponding prediction patches\\
\\
There are two goals with my loss function $L$\\
1. Have values of $L$ correspond well to visual inspection of target and prediction maps in the same way EMD did that for target and prediction patches.\\
2. Use a subset of $P$ to calculate $L$ within a reasonable bound\\
\\
Let $L_k$ be a random variable corresponding to calculating $L$ for a random subset of $P$ of size $k$ and I will define $L=L_N$\\
\\
To accomplish goal 2, I will pick an $\epsilon$ and see how large $k$ must be in order for
\[
(1) Pr( |L_k-L| > \epsilon) = 0
\]
The hope is that this is satisfied with a $k$ not too large, thus only $k$ EMD calculations would need to be made.\\
\\
Let $f(q_i,\hat{q}_i)$ be the EMD between patch $q_i$ and $\hat{q}_i$\\
Let $x$ and $\hat{x}$ be individual precipitation values in the patches\\
Let $g(q_i,\hat{q}_i)$ be $min(\sum_{x \in q_i}{x},\sum_{\hat{x} \in \hat{q}_i}{\hat{x}})$\\
\\
Let $I'(k)$ be the indices for a random subset of $P$ of size $k$. My $L_k$ was as follows:
\[
L_k = \frac{1}{k} \sum_{i \in I'(k)}{f(q_i,\hat{q}_i)}
\]
However $k$ needed to be near $N$ in order satisfy (1). I thus want to try a new loss function which is closer to what an EMD approximation over the whole image would be
\[
L_k = \frac{\sum_{i \in I'(k)}{f(q_i,\hat{q}_i)g(q_i,\hat{q}_i)}}{\sum_{i \in I'(k)}{g(q_i,\hat{q}_i)}} 
\]


\end{document}
